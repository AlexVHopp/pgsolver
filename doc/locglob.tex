\section{Local vs.\ Global Solvers}

The problem of \emph{solving} a parity game $G = (V,V_0,V_1,E,\Omega)$ is, roughly speaking, to determine
which of the players has a winning strategy for that game. However, this does not take starting nodes into
account. In order to obtain a well-defined problem, this is refined in two ways.

The problem of solving the parity game $G$ \emph{globally} is to determine, for \emph{every} node $v \in V$
whether $v \in W_0$ or $v \in W_1$. The problem of solving $G$ \emph{locally} is to determine for a
\emph{given} $v \in V$ whether $v \in W_0$ or $v \in W_1$ holds. In addition, we require a \emph{solver}
to compute (partial) winning strategies in the following sense: a local solver should return one strategy
that is a winning strategy for player $i$ if $v \in W_i$ for the given input node $v$. A global solver
should return two strategies, one for each player s.t.\ player $i$ wins exactly on the nodes $v \in W_i$
if he/she plays according to the strategy computed for her.

Clearly, the local and global problem are interreducible, the global one solves the local one for free,
and the global one is solved by calling the local one $|V|$ many times. But neither of these indirect
methods is particularly clever. Thus, there are algorithms for the global, and other algorithms for the
local problem. We focus on the global problem here, but also use local solvers. In that case we expect
them to determine for \emph{at least} the given node $v$ to which winning set it belongs but possibly and
preferably also for other nodes. These methods can then be used to solve the global problem by calling
the local algorithm \emph{at most} $|V|$ many times.

Suppose $V = \{v_0,\ldots,v_n\}$. Then one can solve $G$ globally using a local solver $A$ as follows.
Start $A$ with starting node $v_0$. It will return two (possibly empty) winning sets $W'_0$ and $W'_1$.
In any case, we will have $W'_0 \cap W'_1 = \emptyset$, but not necessarily $W'_0 \cup W'_1 = V$. Let
$W' := V \setminus (\attr{}{0}{W'_0} \cup \attr{}{1}{W'_1})$. Then taking out all nodes in $W'$ from $G$
will result in a total parity game again by two applications of Lemma~\ref{lem:minusattr} and the fact
that $\attr{}{0}{W'_0} \cap \attr{}{1}{W'_1} = \emptyset$. Let $G'$ be this game. Then the winning positions
for player $i$ in $G$ are $\attr{}{i}{W'_i}$ plus the winning positions for him/her in $G'$. Since $G'$
will be smaller than $G$ this can be used iteratively or recursively, to compute the entire $W_0$ and $W_1$.

The winning strategies can equally be assembled to a winning strategy $\sigma$ for player $i$. Let
$\sigma'$ be player $i$'s winning strategy on the subgame $G'$, $\alpha$ be his/her attractor strategy
for reaching $W'_i$, and $\eta$ be the strategy that guarantees him/her to win on $W'_i$. Then define
\begin{displaymath}
\sigma(v) \enspace := \enspace
\begin{cases}
\eta(v) \enspace, &\mbox{if } v \in W'_i \\
\alpha(v) \enspace, &\mbox{if } v \in \attr{}{i}{W'_i} \setminus W'_i \\
\sigma'(v) \enspace, &\mbox{if } v \mbox{ is a node in } G' \\
\mbox{anything} & \mbox{otherwise}
\end{cases}
\end{displaymath}
The following theorem provides correctness of this construction. 

\begin{proposition}
Let $G = (V,V_0,V_1,E,\Omega)$ be a parity game, let $W_i$ be the winning set of player $i$ and 
$i \in \{0,1\}$. Player $i$ has a positional strategy $\sigma$ for $G$ s.t. $\sigma$ is a positional 
winning strategy for $G$ starting in any node $v \in W_i$.
\end{proposition}

To see that this holds let $\sigma_v$ be positional winning strategies for $G$ starting in $v$ for each node 
$v \in W_i$ (which have to exist by the former theorem). We assume all nodes $v \in W_i$ to be ordered 
w.r.t.\ a well-ordering $<$. A positional strategy $\sigma$ winning from each node $v \in W_i$ can be 
defined as follows:
\begin{displaymath}
\sigma: v \in W_i \cap V_i \mapsto \sigma_{\min_{<} \{u \in W_i \mid v \in dom(\sigma_u)\}} (v)
\end{displaymath}
To see that $\sigma$ indeed is a positional winning strategy for $G$ starting in any node $v \in W_i$ let 
$v_0 v_1 \ldots$ be a $\sigma$-conforming play with $v_j \in W_i$ for all $j \in \Nat$. Since $<$ is a 
well-ordering $\sigma$ finally simulates the strategy of $\sigma_u$ for some $u \in W_i$ and thus player 
$i$ is wins this play.





%%% Local Variables:
%%% mode: latex
%%% TeX-master: "main"
%%% End:
