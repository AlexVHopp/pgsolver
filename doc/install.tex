\section{Installation Guide}

\subsection{Obtaining the Relevant Parts}

You can obtain the source code for \pgsolver from
\begin{center}
    \url{https://github.com/tcsprojects/pgsolver}
\end{center}
Download the latest sources.
\begin{verbatim}
    ~> git clone https://github.com/tcsprojects/pgsolver
\end{verbatim}
This will create a directory \texttt{pgsolver} and various subdirectories in it. Get the required submodules.
\begin{verbatim}
    ~> cd pgsolver
    ~/pgsolver> git submodule update --init
\end{verbatim}

In order to compile \pgsolver from source code you will need the OCaml compiler and the compilation tool \texttt{ocamlbuild}. A convenient way to get
both is to use the OCaml Package Manager \texttt{opam}. Install it via any package manager for your system or download it from
\begin{center}
\url{https://opam.ocaml.org/}
\end{center}
Get the OCaml compiler and Ocamlfind.
\begin{verbatim}
    ~> opam switch 4.03.0 
    ~> eval `opam config env`
    ~> opam install ocamlbuild
\end{verbatim}

If you intent to contribute to the development of \pgsolver you may want to use unit tests as well. This requires two more packages.
\begin{verbatim}
    ~> opam install ocamlfind ounit  
\end{verbatim}
  
\pgsolver consists of many executables; their creation is still guided by \texttt{make}, so you need to make sure that GNU Make
is installed. We recommend version 3.81 or higher but earlier ones, as well as other implementations may suffice as well.

If you want to use any of the reductions to SAT in order to solve parity games then you need one of more SAT solvers. Note that
There are two SAT solver backends that are already supported by
      \pgsolver:

	\begin{itemize}
      %   	\item the SAT solver \texttt{zChaff} developed in Princeton. It is available from
      %   \begin{verbatim}
      % http://www.princeton.edu/~chaff/zchaff.html\end{verbatim}
      %   We recommend version 2007.3.12. Earlier version will probably not suffice.

        \item \texttt{MiniSat} developed in G\"oteborg. It is available from
        \begin{verbatim}
      http://minisat.se\end{verbatim}
        We recommend the C-version of 1.14. Other version will probably not work with our interface.

        \item \texttt{PicoSAT} developed in Linz. It is available from
        \begin{verbatim}
      http://fmv.jku.at/picosat\end{verbatim}
        We recommend version 632. Earlier version will probably not suffice.

    \end{itemize}

% \item (optionally) a C++ compiler like \texttt{g++} for example. This is only necessary
%       to compile \texttt{zChaff} and the parts of \pgsolver that link it. You can get it
%       from
%       \begin{verbatim}
%     http://gcc.gnu.org\end{verbatim}
%\end{itemize}


\subsection{Compiling \pgsolver}

Now change into the \pgsolver directory.
\begin{verbatim}
    ~> cd pgsolver
\end{verbatim}
% The first (and probably most important) step consists of adjusting the \texttt{Config} files.
% There are two config files, \texttt{./Config.default} and \texttt{./satsolvers/Config.default},
% that have to be edited. It is highly recommended to create a copy of both files with the name \texttt{Config}
% in the respective directory that is to be edited instead of the original versions. The Makefile
% checks whether a customized configuration file named \texttt{Config} exists and if so it is used
% instead of the default versions.

% Both configuration files start with declarations about where to find all the programs necessary to build
% the executable \pgsolver.
% \begin{verbatim}
%     OCAMLOPT=ocamlopt
%     OCAMLLEX=ocamllex
%     OCAMLYACC=ocamlyacc
%     CPP=g++
%     OCAMLOPTCPP=g++
% \end{verbatim}
% Change these lines to point to the full path in which the OCaml compiler, lexer and parser
% generator are installed unless they are in the current PATH. The lines pointing to the C++
% compiler only need to be configured if there is at least one sat solver that is to be linked
% with \pgsolver.

% You need to give the full path of you OCaml installation directory.
% \begin{verbatim}
%     OCAML_DIR=/usr/lib/ocaml
% \end{verbatim}


% For each supported SAT solver, there are some lines that need to be configured in order
% to use the include the respective SAT solver.
% \begin{verbatim}
%     WITH_RESPECTIVE_SAT_SOLVER=YES
%     PATH_TO_OBJECT_FILE=...
% \end{verbatim}
% If you do want to have support for the reductions to SAT then make sure that at least one
% SAT solver is properly configured and enabled.

% Once you have adjusted both \texttt{Config} files accordingly you can now compile \pgsolver
% by simply calling the make program.


\begin{verbatim}
    ~/pgsolver> make
\end{verbatim}
This is the same as
\begin{verbatim}
    ~/pgsolver> make all
\end{verbatim}
which is an abbreviation for \verb# make pgsolver generators tools#. If you only want the executable \pgsolver then 
\begin{verbatim}
    ~/pgsolver> make pgsolver
\end{verbatim}
suffices. The same holds for the executables containing benchmark generators and some tools to manipulate parity games.
After successful compilation, the executables can be found in the
subdirectory \texttt{bin}.

You can delete all files that have been created during the compilation process by running
\begin{verbatim}
    ~/pgsolver> make clean
\end{verbatim}

% If you also want to have executable programs that create the benchmarks described in the next
% chapter as well as some possibly useful tools that massage parity games then run
% \begin{verbatim}
%     ~/pgsolver> make generators
% \end{verbatim}
% and
% \begin{verbatim}
%     ~/pgsolver> make tools
% \end{verbatim}
% Finally,
% \begin{verbatim}
%     ~/pgsolver> make all
% \end{verbatim}
% is a synonym for \verb#make pgsolver; make generators; make tools#.


\subsection{Integrating SAT solvers}

\pgsolver currently supports two backend SAT solvers, namely \texttt{PicoSAT} and
\texttt{MiniSat}. It should be also possible to integrate other SAT solvers; see the Developer's Guide for
more information on that subject. In order to integrate one of them into \pgsolver, you need to follow these
steps:\TODO{ML: this needs to be updated}
\begin{enumerate}
\item Download and compile the respective SAT solver.
\item Adjust the \texttt{satsolvers/Config} file s.t.\ the usage of the respective SAT solver is enabled and all required links point to the correct target.
\item Remove the current compilation of \pgsolver: \verb#make cleanall#.
\item Recompile \pgsolver: \verb#make all#.
\end{enumerate}


% \subsubsection{Compiling \texttt{zChaff}}

% Obtain the source code of \texttt{zChaff} (we recommend at least version 2007.3.12), unpack it and
% consult the included \texttt{README} file for instructions on how to compile it. If you are lucky
% then a simple
% \begin{verbatim}
%     ~/zchaff> make
% \end{verbatim}
% will do the job.

% % \hide{
% % Check in the included \texttt{Makefile} or via the output produced during compilation which C++ compiler
% % is being used. For example, in \texttt{Makefile} there should be a line of the form
% % \begin{verbatim}
% %     CC = g++ -Wall
% % \end{verbatim}
% % which tells you that the compiler used to build \texttt{zChaff} is \texttt{g++}. You will have to make
% % sure that this is also the value of the variable \texttt{CPP} in \pgsolver's \texttt{Makefile}. Otherwise
% % \texttt{zChaff} cannot be linked with the modules of the \pgsolver library.
% % }

% The compilation of \texttt{zChaff} produces an executable \texttt{zchaff}. However, \pgsolver uses the
% library version which can be linked into the program directly. This is usually called \texttt{libsat.a}
% and is also produced by \texttt{zChaff}'s compilation process. Make sure that the variable
% \verb#ZCHAFFLIB# in \pgsolver's \texttt{satsolvers/Config} file points to the directory in which \texttt{libsat.a}
% can be found, for example
% \begin{verbatim}
%     #######
%     # ZCHAFF
%     #######
%     ZCHAFFLIB=/usr/local/lib/zchaff-2007-3-12/libsat.a
%     WITH_ZCHAFF=YES
% \end{verbatim}


\subsubsection{Compiling \texttt{PicoSAT}}

Obtain the source code of \texttt{PicoSAT} (we recommend at least version 632), unpack it and
consult the included \texttt{README} file for instructions on how to compile it. Usually,
\begin{verbatim}
    ~/picosat> ./configure && make
\end{verbatim}
will do the job.

The compilation of \texttt{PicoSAT} produces an executable \texttt{picosat}. However, \pgsolver uses the
object files which can be linked into the program directly. Make sure that the variable
\verb#PICOSATDIR# in \pgsolver's \texttt{satsolvers/Config} file points to the directory in which the object files
can be found, for example
\begin{verbatim}
    #######
    # PICOSAT
    #######
    PICOSATDIR=/usr/local/lib/picosat-632
    WITH_PICOSAT=YES
\end{verbatim}


\subsubsection{Compiling \texttt{MiniSat}}

Obtain the source code of \texttt{MiniSat} (we recommend the C-version 1.14), unpack it and
consult the included \texttt{README} file for instructions on how to compile it. Usually,
\begin{verbatim}
    ~/minisat/core> make
\end{verbatim}
will do the job.

The compilation of \texttt{MiniSat} produces an executable \texttt{minisat}. However, \pgsolver uses the
object files which can be linked into the program directly. Make sure that both
\verb#MINISAT# variables in \pgsolver's \texttt{satsolvers/Config} file point to the respective directories, for example
\begin{verbatim}
    #######
    # MINISAT
    #######
    MINISATDIR=/usr/local/lib/minisat/core
    MINISATMTL=/usr/local/lib/minisat/mtl
    WITH_MINISAT=YES
\end{verbatim}



%%% Local Variables:
%%% mode: latex
%%% TeX-master: "main"
%%% End:
