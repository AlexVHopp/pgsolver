\section{Universal Optimisations}
\label{sec:universal}

There are some optimisations that apply to \emph{all} solvers. These universal optimisations efficiently
try to reduce the overall complexity of a given parity game in order to reduce the effort spent by any solver. 
Clearly, such optimisations have to ensure that a solution of the modified game can be effectively and 
efficiently translated back into a valid solution of the original game.

In the following we describe optimisations that are implemented on top of every solving algrithm:
SCC decomposition, detection of special cases, and compression. The next section then describes a generic
algorithm that uses (some of -- depending on the configuration) these optimisations in order to call
a real solver on as few and little parts of a game as possible.   


\subsection{SCC Decomposition}

Let $G = (V, V_0, V_1, E, \Omega)$ be a parity game. A \emph{strongly connected component} (SCC) is a non-empty set 
$S \subseteq V$ with the property that every node in $S$ can reach every other node in $S$, i.e. $uE^*v$ for 
all $u, v \in S$ (where $E^*$ denotes the transitive-reflexive closure of $E$). A strongly connected component 
$S$ is \emph{proper} iff $uE^+v$ for all $u, v \in S$ (where $E^+$ denotes the transitive closure of $E$). In 
other words: An SCC $S$ is proper iff $|S| > 1$ or $S = \{u\}$ and $uEu$.

\begin{theorem}[\cite{tarjan:146}]
Every parity game $G = (V,V_0,V_1,E,\Omega)$ can, in time $\mathcal{O}(|E|)$, be partitioned into SCCs 
$S_0, ..., S_n$ with $V = \bigcup_{i \leq n} S_i$ and $S_i \cap S_j = \emptyset$ for all $i \not= j$.
\end{theorem}

Additionally there is a strict partial ordering $\rightarrow$ on these SCCs which is defined as follows:
\begin{displaymath}
S_i \rightarrow S_j \quad:\iff\quad i \not= j \wedge \exists u \in S_i,\, v \in S_j:\, uEv
\end{displaymath}
This strict partial ordering is generally known as the \emph{topology} of the SCC decomposition. An SCC $S$
is called \emph{final} w.r.t.\ $\to$ if there is no SCC $T$ s.t.\ $S \to T$. Note that every SCC topology
of a finite graph must have at least one final SCC.

SCC decomposition of parity games as a universal optimisation works as follows. First, the game is decomposed 
into SCCs along with the computation of the strict partial ordering $\rightarrow$. Then, all final SCCs 
with respect to $\rightarrow$ are solved by a parity game solver. Since these SCCs are not connected to any 
other SCCs, all solutions obtained in this manner can be directly used as solutions in the global game.

Second, the attractors for both players with respect to the computed winning sets of all maximal solved SCCs 
are computed and removed from the game. The remainder is still a game, but because of the removal some of the
original SCCs may not be SCCs anymore. All ``damaged'' ex-SCCs are again decomposed into SCCs and replaced by 
the new respective decomposition. 
%(Note: There is no need to actually replace the SCC in the data structure; it is more convenient to implement
% this by recursion).
In this way, the remaining decomposition can be used again to solve the rest of the game, again starting
with those SCCs that are now final. 
%Note that it is convenient to automatically solve improper SCCs since otherwise they usually need to be 
%treated differently than normal SCCs in the real solving algorithms.


%\subsection{Excision of Dominions}

%Although there is no obvious way to further reduce a parity game to smaller sub games that need to be solved, 

With this SCC decomposition it is not necessary to require solvers to solve an entire SCC let alone an entire
game. Instead it suffices to have them solve at least a dominion for one of the players. Given an SCC $S$ and
two dominions $D_0, D_1 \subseteq S$ with $D_0 \subseteq W_0$ and $D_1 \subseteq W_1$, one simply computes the 
attractors $A_i := \attr{}{i}{D_i}$ and considers the induced subgame $(S \setminus A_0) \setminus A_1$ which 
can be recursively solved by decomposition into SCCs etc.

% The excision of dominions is particularly useful for algorithms using a certain heuristic to generate ``good'' 
% strategies or for algorithms trying to find small dominions locally; some solution algorithms of the following 
% subsection will apply the dominion excision approach. Since the excision of a dominion possibly leads to a new 
% subgame that can be optimised and decomposed again, this optimisation can be very powerful.\TODO{Sehr schwammiger
% Paragraph.}


\subsection{Detection of Special Cases}

There are certain kinds of special games that can be solved very efficiently by the following procedures. 
W.l.o.g.\ we can assume games to be proper strongly connected components. Remember that SCCs which are not proper
consist of a single node $v$ only that does not have an edge back to itself. The winner of $v$ is the owner
iff there is a successor that he/she wins. Since all successors belong to topologically greater SCCs we can
assume them to be solved already, and thus, the winner of $v$ is easily determined.

\begin{itemize}
\item \emph{Self-cycle games}: Suppose there is a node $v$ such that $vEv$. Then there are two cases depending on
the node's owner $p$ and the parity of the node's priority. If $\Omega(v) \not\equiv_2 p$ then taking the edge 
$(v,v)$ is always a bad choice for player $p$ and this edge can be removed from the game for as long as 
totality is preserved. If $\Omega(v) \equiv_2 p$ then taking this edge is always good in the sense that 
$\{v\}$ is a dominion for player $p$. Hence, its attractor can be removed as described above.

\item \emph{One-parity games}: If all nodes in a proper SCC have the same parity, the whole game is obviously won 
by the corresponding player no matter which transition the player uses. Hence, a winning strategy can be found 
by random choice.

\item \emph{One-player games}: A game $G$ is a one-player game for player $i$ iff for all $v \in V_{1 - i}$ we have
$|vE| = 1$. Such a one-player game that is an SCC can be solved using a simple fixed-point iteration. 
%\begin{align*}
%E_0\,\, &:= E \\
%E_{j + 1} &:= \{(u, v) \mid uE_jv \vee \exists w:(uE_jw \wedge wE_jv \wedge \Omega(u) \geq \Omega(w) \wedge \Omega(u) \equiv_2 i )\} \\
%E'\,\, &:= \bigcup_{j \in \Nat} E_j
%\end{align*}
%Now the following holds: 
Player $i$ wins the game iff there is a node $u$ with $\Omega(u) \equiv_2 i$ and $u$ is reachable from itself
on a path that does not contain a priority greater than $\Omega(u)$.
% As player $i$ wins the game iff there is a cycle in the game won by player $i$ the fixed point iteration 
%simply computes which nodes having parity $i$ can reach itself without seeing a greater priority than their 
%own. Thus, if $E'$ connects $i$-parity nodes with itself, there needs to be a cycle using that particular 
%node being won by player $i$. The conversion also holds: If there is no $i$-parity node $u$ s.t. $uE'u$ then 
%the whole SCC is won by player $1-i$.
If there is such a cycle won by player $i$ then the rest of the SCC lies in the attractor of the cycle (since 
player $i$ is the only one to make choices); otherwise, if there is no cycle won by player $i$, the whole game 
is won by player $1-i$.
\end{itemize}


\subsection{Priority Compression}

The complexity of a parity game rises with the number of different priorities in the game. This optimisation step
attempts to reduce this number. Note that it is not the actual values of priorities that determine the winner.
It is rather their \emph{parity} on the one hand and their \emph{ordering} on the other. For instance, if there are 
two priorities $p_1 < p_2$ in a game with $p_1 \equiv_2 p_2$ but there is no $p'$ such that $p_1 < p' < p_2$ and 
$p' \not\equiv_2 p_1$ then every occurrence of $p_2$ can be replaced by $p_1$. 

In general, let $P = (p_0,\ldots,p_k)$ be the list of all the priorities occurring in a game $G = (V,V_0,V_1,E,\Omega)$ 
s.t.\ $p_{i-1} < p_i$ for all $0 \le i < k$. W.l.o.g.\ we assume $p_0$ to be even. If the least priority occurring in
$G$ is odd, then simply add $p_0 = 0$ to this list which does not affect the construction in any way. We will also
use $P$ to denote the \emph{set} of all elements in $P$.

Take a decomposition of $P$ into maximal sublists of elements with the same parity, i.e.\ 
\begin{displaymath}
P \enspace = \enspace (p_{0,0},\ldots,p_{0,m_0},p_{1,0},\ldots,p_{1,m_1},\ldots,p_{n,0},\ldots,p_{n,m_n})
\end{displaymath}
with $p_{i,j} \equiv_2 p_{i,j'}$ for all $0 \le i \le n$, $0 \le j < j' \le m_i$ and 
$p_{i,m_i} \not\equiv_2 p_{i+1,0}$ for all $0 \le i < n$. This defines a partial mapping $\omega: \Nat \to \Nat$ as 
\begin{displaymath}
\omega(p) \enspace = \enspace 
\begin{cases} 
i &, \mbox{if } p = p_{i,j} \mbox{ for some } j  \\
\mathrm{undefined} &, \mbox{otherwise}
\end{cases}
\end{displaymath}
Note the following facts about $\omega$:
\begin{itemize}
\item $\omega$ is defined on all priorities occurring in $G$;
\item $\omega$ is \emph{decreasing}: we have $\omega(p) \le p$ for all $p \in P$;
\item $\omega$ is \emph{monotone}: for all $p,p'$ with $p \le p'$ we have $\omega(p) \le \omega(p')$;
\item $\omega$ \emph{preserves parities}: we have $\omega(p) \equiv_2 p$ for all $p \in P$;
\item $\omega$ is \emph{dense}: for all $p,p' \in P$ with $\omega(p)+1 < \omega(p')$ there is a $p''$ with 
$\omega(p) < \omega(p'') < \omega(p')$.
\end{itemize} 
Now define another parity game $G' := (V,V_0,V_1,E,\Omega')$ with $\Omega'(v) := \omega(\Omega(v))$. I.e.\ $G'$
results from $G$ by reducing the priorities according to the function $\omega$. Then $G'$ is equivalent to $G$
in the sense that the players' winning regions coincide in the two games, and a winning strategy $\sigma$ for
some player $i$ in $G$ is also a winning strategy for him/her in $G'$ and vice-versa. This is guaranteed by the
properties of $\omega$ identified above: due to monotonicity and preservation of parities, the greatest priority
ocurring in an infinite play in $G$ is even iff it is even in the same play in $G'$. The property of being
decreasing guarantees that this should in general be an optimisation, and density says that $G'$ is optimal 
w.r.t.\ this optimisation. 
 
% \begin{lemma}
% Every parity game $G = (V,V_0,V_1,E,\Omega)$ induces an equivalent parity game $G' = (V,V_0,V_1,E,\Omega')$ where 
% $\Omega'$ is inductively defined as follows:
% \begin{displaymath}
% \Omega'(v) \enspace := \enspace
% \begin{cases}
% m(v) \enspace, &\mbox{if } \Omega(v) \equiv_2 m(v) \\
% m(v) + 1 & \mbox{otherwise}
% \end{cases}
% \end{displaymath}
% and
% \begin{displaymath}
% m(v) \enspace := \enspace \max \{0, \Omega'(w) \mid \Omega(w) < \Omega(v)\}
% \end{displaymath}
% Then $W_0(G) = W_0(G')$, $W_1(G) = W_1(G')$ and every winning strategy for player $i \in \{0, 1\}$ w.r.t. $G$ is a winning strategy for player $i$ w.r.t. $G'$ and vice versa.
% \end{lemma}

% Since $\Omega(v) \leq \Omega(w)$ iff $\Omega'(v) \leq \Omega'(w)$ as well as $\Omega(v) \equiv_2 \Omega'(v)$ it is not hard to see that the greatest priority occurring infinitely often in a play w.r.t. $G$ has the same parity as w.r.t. $G'$.



\subsection{Priority Propagation}

Note that any play visiting a node $v$ has to -- due to totality -- also visit one of the successors of 
$v$. Now suppose that the priorities of all successors of $v$ are greater than the priority of $v$ itself.
Then $v$'s priority is irrelevant in the sense that no play is won by either player because $v$'s priority
occurs in it. For those plays not visiting $v$ at all this is trivial and for those plays that do visit
$v$ this is simply because $v$ is certainly not the geatest priority occurring in this play, let alone
occurring infinitely often. Hence, $v$'s priority can be replaced by a greater one. 

In general, let $G = (V,V_0,V_1,E,\Omega)$. Applying \emph{backwards propagation} in node $v$ results in
the game $G' = (V,V_0,V_1,E,\Omega')$ where 
$\Omega' = \Omega[v \mapsto \max \{ \Omega(v), \min \{ \Omega(w) \mid w \in vE \} \}]$.
Similarly, \emph{forwards propagation} replaces $v$'s priority with the minimum of all
priorities of its predecessors if that is greater than its current priority. It is equally sound because
any play that visits $v$ infinitely often must visit one of its predecessors infinitely often too.

Both backwards and forwards propagation can be iterated and combined thus reducing the range of priorities
in a game.






%%% Local Variables:
%%% mode: latex
%%% TeX-master: "main"
%%% End:
