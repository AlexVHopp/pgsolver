\section{Running \pgsolver}

\subsection{Invocation}

A successful compilation produces an executable binary called \texttt{pgsolver} which, unless
specified otherwise in \texttt{Makefile}, resides in the subdirectory \texttt{bin}. Type
\begin{verbatim}
    ~/pgsolver> bin/pgsolver --help
\end{verbatim}
or
\begin{verbatim}
    ~/pgsolver> bin/pgsolver -help
\end{verbatim}
for a list of command-line options and a description of the command-line parameters that the
program expects.

The specification of the parity game to be solved -- cf.\ Sect.~\ref{sec:specs} -- can either be
given in a file or through \texttt{stdin}. The latter case is the default, and to switch to the
former one only needs to give the name of the file including the specification as a command-line
argument.
\begin{verbatim}
    ~/pgsolver> bin/pgsolver tests/test1.gm
\end{verbatim}

\subsection{Command-Line Parameters}

\pgsolver's behaviour can be changed through command-line parameters. In particular, you can determine
the algorithm that is used for solving, specify whether or not you want to view the result in text
mode or graphically, etc. Command-line parameters can be given in any order, although some are
conflicting and in that case the latter overrides the former; for example when you specify more than
one algorithm that should be used for solving. The currently understood command-line parameters are:
\begin{description}
\itemsep3mm
\item[\nonterminal{filename}] \ \\
   Tells \texttt{pgsolver} to look for the specification of the parity game to solve in the file
   \nonterminal{filename}.

\item[{\ttfamily  -v \nonterminal{level}}] \ \\
   Sets the verbosity level, valid arguments are $0$--$3$, the default is 1. With verbosity level $0$,
   \texttt{pgsolver} will be very humble and not bother \texttt{stdout} with its pathetic goobledigook.
   With verbosity level 1, it will tell you what it does and what the winning regions and strategies
   are. With verbosity level 2, it will tell you a bit more, but this very much depends on which
   algorithm is chosen for solving etc. It may for example tell you something it has found out about the
   SCC structure or priority distribution in the game. Verbosity level 3 is for debugging purposes. Again,
   this very much depends on the solving algorithm.

%\item[{\ttfamily --quiet}] \ \\
%   Causes the program to be quiet, same as \texttt{-v 0}.

%\item[{\ttfamily --verbose}] \ \\
   %Causes the program to be verbose, same as \texttt{-v 2}.

%\item[{\ttfamily --debug}] \ \\
   %Causes the program to be very verbose, same as \texttt{-v 3}.

\item[{\ttfamily -d \nonterminal{filename}}] \ \\
   Tells \texttt{pgsolver} to open the file \nonterminal{filename} for writing and print into it the
   \texttt{dot}-code of the parity game as a graph, see also Sect.~\ref{sec:viewing}. If the game has
   been solved during this invocation then the graphical presentation will display the winning regions
   and strategies. You can use option \texttt{-n} in combination with this one to display a pure parity
   game, i.e.\ without the winning information.

% \item[{\ttfamily -f}] \ \\
%    Causes \texttt{pgsolver} to print the whole game on \texttt{stdout} again. The format is the one
%    that \texttt{pgsolver} would expect as input.\TODO{Hat das einen besonderen Sinn?}

% \item[{\ttfamily --sanitycheck}]  \enspace or {\ttfamily -sc} \\
%   Checks whether the given parity game is well-formed, meaning that it particularly does not contain any edge leading to inaccessible nodes.

% \item[{\ttfamily -n}] \ \\
%    Tells \texttt{pgsolver} to suppress solving. This can be used to simply draw a parity game in
%    \texttt{dot}-format without the information about winning regions and strategies. This is a default
%    selection, i.e.\ \pgsolver will not solve a game unless you specify explicitly which algorithm it
%    should use. See below for the command-line parameters that let you do so.

\item[{\ttfamily --disableglobalopt}]  \enspace or {\ttfamily -dgo} \\
   Tells \texttt{pgsolver} to completely disable any global optimisations. Global optimisation is enabled by default.

% \item[{\ttfamily --disableuselesscycles}]  \enspace or {\ttfamily -dul} \\
%    Tells \texttt{pgsolver} to disable to automatic deletion of useless self cycles. This is a global optimisation and is enabled by default.
% 
% \item[{\ttfamily --disableusefulcycles}]  \enspace or {\ttfamily -duf} \\
%    Tells \texttt{pgsolver} to disablee the automatic utilisation of useful self cycles. This is a global optimisation and is enabled by default.

\item[{\ttfamily --disablesccdecomposition}]  \enspace or {\ttfamily -dsd} \\
   Tells \texttt{pgsolver} to decompose the game into strongly connected components. This is enabled by default.

\item[{\ttfamily --disablelocalopt}]  \enspace or {\ttfamily -dlo} \\
   Tells \texttt{pgsolver} to completely disable any local optimisations. Local optimisation is enabled by default.

% \item[{\ttfamily --enableprioprop}]  \enspace or {\ttfamily -pp} \\
%    Tells \texttt{pgsolver} to enable priority propagation. This is a local optimisation and is disabled (!) by default.
% 
% \item[{\ttfamily --disablepriocomp}]  \enspace or {\ttfamily -dcp} \\
%    Tells \texttt{pgsolver} to disable priority compression. This is a local optimisation and is enabled by default.

\item[{\ttfamily --disablespecialgames}]  \enspace or {\ttfamily -dsg} \\
   Tells \texttt{pgsolver} to completely disable any algorithms solving special instances. This is enabled by default.

% \item[{\ttfamily --disablesingleparity}]  \enspace or {\ttfamily -dpa} \\
%    Tells \texttt{pgsolver} to disable the automatic solving of single parity instances. This is a special instance solving optimisation and is enabled by default.
% 
% \item[{\ttfamily --disablesingleplayer}]  \enspace or {\ttfamily -dpl} \\
%    Tells \texttt{pgsolver} to disable the automatic solving of single player instances. This is a special instance solving optimisation and is enabled by default.

\item[{\ttfamily --verify}]  \enspace or {\ttfamily -ve} \\
   This causes \texttt{pgsolver} to perform, after solving, an additional check that the reported winning
   strategies are correct; useful for finding bugs in new algorithm implementations.

% \item[{\ttfamily --verify2}]  \enspace or {\ttfamily -ve} \\
%    Same as {\ttfamily --verify} but use a different verification method; useful for finding bugs in new
%    algorithm implementations if you have used \texttt{--verify} but want to be really sure.
% 
% \item[{\ttfamily --verify3}]  \enspace or {\ttfamily -ve} \\
%    Still the same as {\ttfamily --verify} or {\ttfamily --verify2} but uses yet again a different verification method;
%    useful for finding bugs in new algorithm implementations if you have used \texttt{--verify} and
%    \texttt{--verify2} but generally have issues with trust.

\item[{\ttfamily --justheatCPU}] \enspace or {\ttfamily -jh}  \\
   On large parity games, printing the winning information can take a long time because of the bottleneck
   \texttt{stdout}. This option turns the printing of that information off. Useful if one is interested in
   running times only for instance.

\item[{\ttfamily --changesat \nonterminal{satsolver}}] \enspace or {\ttfamily -cs \nonterminal{satsolver}}  \\
   Selects the SAT solver backend that is to be used. Only affects parity game solving algorithms that
   depend on SAT solving. If there are less than two SAT solvers linked into \pgsolver, this parameter
   is disabled.

\item[{\ttfamily --printsolonly}] \enspace ${}$ \\
	Instead of printing a human readable form of the solution, \pgsolver prints a
	format that can be directly parsed again by the framework.

\item[{\ttfamily --printsolvedgame}] \enspace or {\ttfamily -pg} \\
	Prints the solved game in a parseable form. This will be a subgame of the
	original parity game in which all nodes where the winner of the node is also
	the owner are restricted to the winning edge.

\item[{\ttfamily --parsesolution \nonterminal{file}}] \enspace or
{\ttfamily -ps \nonterminal{file}}  \\
	Parses a previously printed solution into \pgsolver. This allows you to, for
	instance, verify a particular solution.

\item[{\ttfamily --solverinfo}] \enspace  ${}$ \\
	Prints out a list of all available solvers.

\item[{\ttfamily --globallysolve \nonterminal{solver}}] \enspace or
{\ttfamily -global \nonterminal{solver}}  \\
	Chooses a particular global solver to solve the game at hand.

\item[{\ttfamily --locallysolve \nonterminal{solver} \nonterminal{node}}]
\enspace or {\ttfamily -local \nonterminal{solver} \nonterminal{node}}  \\
	Chooses a particular local solver to solve the game at hand starting from a
	specified node.

\item[{\ttfamily --args \nonterminal{arguments}}] \enspace or
{\ttfamily -x \nonterminal{arguments}}  \\
	Provides additional arguments to a solvers. You can get the full list of
	arguments for a particular solver by calling \pgsolver with {\ttfamily -x
	"help"}.

\item[{\ttfamily --generator \nonterminal{generator} \nonterminal{arguments}}]
\enspace or {\ttfamily -gen \nonterminal{generator} \nonterminal{arguments}}  \\
	Generates a game using one of the included generators directly for \pgsolver to
	solve.


% 
% \item[{\ttfamily --viasat}] \enspace or {\ttfamily -vs} \\
%    Use the small progress measure encoding for propositional logic and a reduction to SAT due to Lange.
%    Also see the command-line parameters that select the SAT solver to be used.
%    This parameter is only available if there is at least one SAT solver linked into \pgsolver.
% 
% \item[{\ttfamily --stratimprove}] \enspace or {\ttfamily -si} \\
%    Use the strategy improvement algorithm due to Jurdzi{\'n}ski and V\"oge.
% 
% \item[{\ttfamily  --optstratimprov}] \enspace or {\ttfamily -os} \\
%    Use the strategy improvement algorithm due to Schewe.
% 
% \item[{\ttfamily --smallprog}] \enspace or {\ttfamily -sp} \\
%    Use the small progress measure algorithm due to Jurdzi{\'n}ski.
% 
% \item[{\ttfamily --satsolve}] \enspace or {\ttfamily -ss} \\
%    Tells \texttt{pgsolver} to encode a direct NP predicate that is to be solved due To Friedmann.
%    This parameter is only available if there is at least one SAT solver linked into \pgsolver.
% 
% \item[{\ttfamily --recursive}] \enspace or {\ttfamily -re} \\
%    Use the recursive algorithm due to Zielonka.
% 
% \item[{\ttfamily --modelchecker}] \enspace or {\ttfamily -mc}  \\
%    Use the $\mu$-calculus model checker due to Stevens and Stirling.
% 
% \item[{\ttfamily --dominiondec}] \enspace or {\ttfamily -dd} \\
%    Use the dominion decomposition algorithm due to Jurdzi{\'n}ski, Paterson, and Zwick.
% 
% \item[{\ttfamily --bigstep}] \enspace or {\ttfamily -bs} \\
%    Use the big-step procedure due to Schewe.
% 
% \item[{\ttfamily --guessstrategy}] \enspace or {\ttfamily -gs} \\
%    Use the strategy guessing heuristic.

\end{description}



\subsection{Output}

An invocation of
\begin{verbatim}
    ~/pgsolver> bin/randomgame 10 10 2 4 | bin/pgsolver -global recursive
\end{verbatim}
will typically create output on \texttt{stdout} like this:
\begin{verbatim}
    Parsing ..................... 0.00 sec
    Chosen solver `recursive' ..... 0.00 sec

    Player 0 wins from nodes:
      {0,2,5,6,7,8}
    with strategy
      [0->6,2->2,5->0,6->2,7->7,8->8]

    Player 1 wins from nodes:
      {1,3,4,9}
    with strategy
      [1->9,3->9,4->3,9->9]
\end{verbatim}
It reports the times it took to parse the input and to solve the game specified in this input
as well as which solver has been used. Then it reports for both players the sets of winning
regions in the game as a list of natural numbers (each node's index in the game) in ascending order.
The positional strategies are reported as a list of pairs of the form $x$\texttt{->}$y$ meaning
that in node $x$, the player at hand should move to node $y$. This list is also sorted in
ascending order w.r.t.\ $x$. Furthermore, the set of nodes given as $x$'s in this list is
exactly the set of nodes belonging to that player.




%%% Local Variables:
%%% mode: latex
%%% TeX-master: "main"
%%% End:
