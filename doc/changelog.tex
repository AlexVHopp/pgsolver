\section{Change Log}

Changes are classified according to their importance. Changes effecting only the performance of
parts of the tool or are less important otherwise are marked with one asterisk ($\ast$). Changes
that are worth noting like the addition of new features etc.\ are marked with two asterisks 
($\ast\ast$). Finally, major changes that would effect usability, for example breaks in backwards
compatibility, changes in the user interface, etc., are marked with three asterisks ($\ast\ast\ast$).

Changes incorporated into version 4.0:
\begin{itemize}
\item[$\ast\ast\ast$] The interface and implementation of the module \texttt{Paritygame} have undergone some major changes,
     mostly concerning the data types for parity games. This type is now abstract, so access to and manipulations of parity games
     are now only possible through specialised functions in this module. A direct access of parity games as arrays is no longer
     possible. Consequently, access to parity games in all other modules has been changed accordingly. This encapsulation step 
     enables us to change the internal data structures for parity games easily for optimisation purposes without changing any of 
     the algorithms around them. It is also worth noticing that a node in a parity game now knows its predecessors as well as its
     successors. Hence, functions for computing transposed graphs etc.\ have been removed.
\item[$\ast\ast$] \pgsolver is now built using \texttt{ocamlbuild}. It requires the OCaml packages \texttt{ocamlfind} and \texttt{ounit}.
     Get them using \texttt{opam}. Moreover, \pgsolver now requires OCaml version 4.03.0.
\item[$\ast\ast\ast$] If a textual presentation of a parity game starts with the keyword \texttt{parity} then it is followed by the number of
     nodes in the game, rather than the largest identifier of a node in the game. So any ``\texttt{parity} $n$'' should now be 
     ``\texttt{parity} $n+1$''.
\item[$\ast$] A framework for the easy creation of benchmarks derived from model-checking problems is now provided. 
\item[$\ast\ast$] Support for the \texttt{Z3} has been discontinued.  
\end{itemize}

Changes incorporated into version 3:
\begin{itemize}
\item[$\ast\ast$] \pgsolver is now linkable as library.
\item[$\ast$] A new local strategy improvement algorithm has been added.
\item[$\ast\ast$] Local Solving of Games has been enabled.
\item[$\ast$] Generators can now be linked directly into \pgsolver.
\item[$\ast$] Two new randomized strategy improvement algorithms have been incorporated.
\item[$\ast$] The lists of solvers and generators are now maintained in \texttt{./Solvers} and \texttt{./Generators} respectively.
\item[$\ast$] Printing and parsing of solution and strategies.
\item[$\ast$] Conversion between min-parity and max-parity games as a new transformer feature.
\item[$\ast$] Two new generators: Towers of Hanoi as a reachability game and a fairness verification of an elevator system.
\end{itemize} 



Changes incorporated into version 2:
\begin{itemize}
\item[$\ast$] In order to allow more efficient parsing, the specification format for parity
      games has been extended. It is now possible to include at the beginning of the specification
      the maximal index of a node in the game. If this is done, then parsing will be quicker.
\item[$\ast$] All the random game generators are now based on a more efficient generation
      of sets of random numbers.
\item[$\ast$] The computation of attractor regions has been improved.
\item[$\ast\ast\ast$] Confusing terminology has been clarified: the algorithm due to Stevens and
     Stirling \cite{StevensStirling98} is now referred to as the \emph{model checker} rather than the 
     former \emph{game-based algorithm}. Command line parameters to \texttt{pgsolver} have been 
     changed accordingly.
\item[$\ast\ast$] A useful benchmarking tool has been included in the distribution.
\item[$\ast$] This change log has been included in this documentation -- in case you hadn't noticed.
\end{itemize} 

 

%%% Local Variables: 
%%% mode: latex
%%% TeX-master: "main"
%%% End: 
