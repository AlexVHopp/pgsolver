\label{chp:devguide}

The purpose of this chapter is to provide enough insight into the structure
of the \pgsolver tool such that it becomes possible to extend it with a new algorithm.
We assume reasonable OCaml programming skills and some familiarity with parity games, though.

Implementing a new algorithm requires two steps only.
\begin{enumerate}
\item Create an OCaml source code file that contains the solver. This is just a function
      which takes a parity game and returns a partition of (a subset of) its node set into
      winning regions and winning strategies for the two players. It must reside in a new
      module.
\item Integrate this function into the main program such that the new solver can be used
      by giving \texttt{pgsolver} a certain command-line argument. Then adjust the
      \texttt{SolverList} such that the new module is compiled with the rest of the tool.
\end{enumerate}


\section{Structure of the Source Code}

In the following, all references to files are given relatively to the directory into which
\pgsolver was unpacked during the installation process. There, the subdirectory \texttt{src}
is the important one for this step. It contains the source files of all the modules that
make up the \pgsolver library. They are organised into the following subdirectories respectively.
Note that, when we speak of \emph{modules} we often refer to files only that implicitly count as
a module in OCaml.
\begin{description}
\item[\texttt{generators}] contains modules that make up the benchmark generators;
\item[\texttt{tools}] contains tools for the creation or manipulation of parity games;
\item[\texttt{paritygame}] contains modules that define parity games as a data structure and
     provide functionality for that, e.g.\ in the form of procedures that compute SCC decompositions
     etc.;
% \item[\texttt{parser}] contains modules for the lexer and parser of the parity game specification
%     format;
\item[\texttt{pgsolver}] contains modules for the actual program \texttt{pgsolver};
\item[\texttt{solvers}] contains the parity game solvers, each in a separate module;
% \item[\texttt{zchaff}] contains the interface to the external C++ functions in the \texttt{zChaff}
%     library.
\end{description}

\section{Definitions of Important Data Types}

The signature file
\begin{verbatim}
src/paritygame/paritygame.mli
\end{verbatim}
contains the definition of parity games as a data structure.
\begin{verbatim}
type paritygame = (int * int * int array * string option) array
\end{verbatim}
A parity game is internally stored as an array of node definitions. Hence, the size of the array
is at least the number of nodes in the game. A node is a value of type \texttt{int} -- it is
implicitly assumed that the nodes' names in a game are $0,\ldots,n$ for some $n \in \Nat$. The
value stored in the array at position $i$ defines the node $i$ in the game. Such a value is of
the form \verb#(priority, player, successors, name)# where
\begin{itemize}
\item \texttt{priority} is a value of type \texttt{int} defining the node's priority. Admissible
      values are $0,1,2,\ldots$
\item \texttt{player} is a value of type \texttt{int} determining whether or not node $i$ belongs
      to $V_0$ or $V_1$. Admissible values are only $0$ or $1$.
\item \texttt{successors} is an array with entries of type \texttt{int} defining the outgoing edges
      of node $i$. There is an edge to node $j$ iff $j$ occurs somewhere in the array. Multiple
      occurrences of successor nodes in such an array are not explicitly forbidden. Hence, it is
      actually possible to define a multi-graph. However, it is easy to see that
      multiple edges do not affect the winning regions or strategies.
\item \texttt{name} of type \texttt{string option} is an optional name for the node which can be
      used in displaying the game. The default value is \texttt{None} representing no name. A value
      of the form \verb#Some("Europe")# declares the string \texttt{Europe} as the symbolic name of
      node $i$.
\end{itemize}

\begin{example}
The parity game in Ex.~\ref{ex:examplegame} whose graph is shown in Fig.~\ref{fig:examplegame} on
page~\pageref{fig:examplegame} would, for example, be represented as the following array.
\begin{verbatim}
[| (6,1,[|4;2|],Some("Africa")); (8,1,[|2;4;3|],Some("America"));
   (7,0,[|3;1;0;4|],Some("Asia")); (6,0,[|4;2|],Some("Australia"));
   (5,1,[|0|],Some("Antarctica")) |]
\end{verbatim}
\end{example}

It is possible to leave nodes undefined by given them the value \verb#(-1,-1,[||],None)# for example. 
% Take the Jurdzi{\'n}ski games shown in Fig.~\ref{fig:marcingame}
% on page~\pageref{fig:marcingame} for example. When generating these it may be helpful to add a fourth
% node to every small triangle that makes up the entire rectangular structure in order to have a simpler
% enumeration of the graph's nodes. 
%Such a node could be defined using .

The signature file
\begin{verbatim}
src/paritygame/paritygame.mli
\end{verbatim}
also contains two type definitions for a parity game's \emph{solution} -- the node set's partition into
winning regions -- and positional \emph{strategies}.
\begin{verbatim}
type solution = int array
type strategy = int array
\end{verbatim}
Internally, both simply are arrays with entries of type \texttt{int}. Note that a solution or a strategy
only makes sense in the context of a parity game, but this is not respected by this definition of data
structures. The program itself has to ensure that it does not lose the connection between a solution array
and the corresponding game for example.

Given a parity game, represented as an array of length $n$ as above, a solution to this is an \texttt{int}
array \texttt{s} of length $n$ as well. The entry at position $i$ in \texttt{s} is either $0$ or $1$ --
indicating whether or not node $i$ belongs to $W_0$ or $W_1$. The value $-1$ is also allowed, enabling partial
solving. This is used to indicate that a solution for node $i$ has not been computed.

An array of type \texttt{strategy} contains, in effect, two positional strategies -- one for each player.
Again, such strategies are only well-defined in the context of a given parity game, although this is
not reflected by the typing. Remember that the node set $V$ of a parity game is partitioned into
$V_0$ and $V_1$, assigning an owner to each node in the game. If there are $n$ nodes, then a valid
strategy array is also of size $n$. Let \texttt{game} be the array representation of a parity game as
described above. Now take some $i < n$ and let \texttt{(priority,player,succs,name)} be the definition of
node $i$, i.e.\ the entry at position $i$ in the array \texttt{game}. Remember that \texttt{succs} is, again,
an array with entries of type \texttt{int} that contains the indices of all the successors of $i$. Now
the array \texttt{strategy} defines two positional strategies $\sigma_0$ and $\sigma_1$ for players
$0$ and $1$ in the following sense. The entry at position $i$ is $j$ iff player $\sigma_p(i) = k$ where
$k$ is the value of the \texttt{succs}-array at position $j$, and $p$ is the value of \texttt{player}.
In other words, the \texttt{strategy} array contains for every node the index of the outgoing edge that
the owner of the node should take in a move.

Clearly, the strategies represented by such arrays need not be winning strategies. It is assumed, though,
that, in the context of a solution array, the strategies are winning strategies on those parts of the game
that are won by each player respectively. Hence, admissible values for entries in a strategy array
are $0,1,\ldots$ but values $j \ge 1$ may lead to exceptions due to array indices that are out of bounds.
The value $-1$ is also allowed to indicate that a strategy is not defined for a certain node. This enables
partial solving and should only occur in conjunction with a $-1$ in the corresponding position of the
solution array indicating that the node has not been solved by a solver.

\section{Implementing a New Solver}

We will exemplarily describe how to implement and integrate a new algorithm that solves parity games. There
is more than one way to do so but there is a unique easiest way which we will follow here.

Suppose you have finally come up with a deterministic polynomial time algorithm. Surely this cannot miss in
the \pgsolver library. It is also quite probable that the algorithm is complex -- for otherwise someone
else would have found it before. Therefore you cannot wait for us to implement it. By the time we have
understood all the algorithm's subtleties our programming skills will have bitten the dust. In that case you
better do it yourself.

Find an expressive name for your algorithm that will be used to create a module and also later a command-line
parameter for the \texttt{pgsolver} program. Say \emph{Deterministic Polynomial Time Solver} was a good name.
Then create two files in the \texttt{src/solvers} subdirectory, for example called
\begin{verbatim}
detpoly.ml
detpoly.mli
\end{verbatim}
The signature file \texttt{detpoly.mli} must look like this.
\begin{verbatim}
val solve : Paritygame.paritygame ->
              Paritygame.solution * Paritygame.strategy
\end{verbatim}
The name of the function is actually irrelevant, but calling it \texttt{solve} is not a bad idea. Its type
is mandatory though. It must take a parity game and return a pair of a solution and a strategy using the
data types described above.

Now sit down and hack the code for the implementation of your solver into the file \texttt{detpoly.ml}. In the
simplest form this will look like
\begin{verbatim}
open Paritygame ;;
let solve game = ...
\end{verbatim}
However, note that this function will be called in the main program, and as the argument \texttt{game} it will
receive the parity game from the input. Hence, this does not automatically make use of the universal solver
described in Sect.~\ref{sec:universal}. This is because one cannot guarantee that the universal optimisations
compression, SCC decomposition and solving of special cases speed up \emph{any} solver.

If your algorithm turns out to be that quick you might not want the universal optimisations in which case you
can skip the following description of how to employ the universal solver with your new algorithm as a backend
and continue with the next section.

So you are still with us -- thanks for the trust you have in our universal solver! Now, you will still have
to implement your algorithm in a function of type
\texttt{Paritygame.paritygame -> Paritygame.solution * Paritygame.strategy} but it may be better to choose
a different name. For instance, create a function
\begin{verbatim}
let my_solver game = ...
\end{verbatim}
in \texttt{detpoly.ml} that contains the implementation of your solver. This may now assume that the argument
\texttt{game} consists of a single SCC only such that the priorities of at least two nodes have different
parities and it is not just one player who has real choices in all the nodes in this game.

Now all you have to do is to make sure that the \texttt{solve}-function declared in \texttt{detpoly.mli}
calls the universal solver using your solver as a backend. This is easily done, \texttt{detpoly.ml} should
look like this.
\begin{verbatim}
let solve = Univsolve.universal_solve
             (Univsolve.universal_solve_init_options_verbose
              !Univsolve.universal_solve_global_options)
              my_solver
\end{verbatim}
Note that, if your solver implemented in \verb#my_solver# is recursive, it can also call \verb#solve# instead
of \verb#my_solver# recursively in order to have the universal optimisations done in every step.


\section{Integrating the New Solver}

Continuing with the example of the previous section we assume that you have implemented a solver in the file
\texttt{src/solvers/detpoly.ml}. In particular, this file contains a function \texttt{solve} of type
\texttt{Paritygame.paritygame -> Paritygame.solution * Paritygame.strategy} which is declared in
\texttt{src/solvers/detpoly.mli}.

You will have to ensure that your file does automatically get compiled when \pgsolver is being built using
the \texttt{make} command. In \texttt{SolverList} you will find a variable declaration listing all the modules
that make up the entire program. It should look like this.
\begin{verbatim}
PGSOLVERSLIST=
	obj/recursive.cmx \
	obj/stratimprovement.cmx \
	...
	obj/stratimprlocal.cmx \
	obj/stratimprdisc.cmx 
\end{verbatim}
Append to this list \texttt{obj/detpoly.cmx}. It need not be at the end of this list but you have to make
sure that it occurs behind any other module that contains code which is used by your solver. This could
be the \texttt{paritygame} and \texttt{univsolve} part for example and will most certainly be the
\texttt{solvers} part. Now running \texttt{make} should compile the \texttt{detpoly}-module as well.

At last, you have to integrate your algorithm into the program so that it can be used through the
\texttt{pgsolver} program when given a certain command-line option. This could not be an easier. Add to
your file \texttt{detpoly.ml} the following code after your \texttt{solve} function.
\begin{verbatim}
let _ = Solvers.register_solver
          solve
          "detpoly"
          "dp"
          "use the brand-new deterministic polytime algorithm"
\end{verbatim}
The function \texttt{register\_solver} from the module \texttt{solvers.ml} does everything for you. You
simply have to give it four arguments.
\begin{itemize}
\item The first one, here \texttt{solve} is the function of the type described above that implements
      your algorithm.
\item The second one is a string which should contain a concise but reasonably expressive synonym for your
      algorithm. It must not contain white spaces because it will be prefixed by two dashes ``\texttt{--}''
      and used as a command-line parameter that tells \texttt{pgsolver} to use your algorithm.
\item The third one is a string which contains an even shorter synonym for your algorithm. It will be
      prefixed by a single dash symbol ``\texttt{-}'' and used as a command-line parameter for \texttt{pgsolver}
      with the same effect.
\item The fourth argument is again a string which contains a description of what \texttt{pgsolver} does
      when given any of the the command-line parameters derived from the second or third argument.
\end{itemize}
After compilation, called \texttt{bin/pgsolver --help} should result in output containing the following
lines.
\begin{verbatim}
PGSolver Collection Ver. 3: Parity Game Solver
Authors: Oliver Friedmann and Martin Lange, University of Munich,
         2008-2010
http://www.tcs.ifi.lmu.de/pgsolver

Usage: pgsolver [options] [infile]
Solves the parity game given in <infile>. If this argument is omitted it
reads a game from STDIN.

Options are
  -v <level>
     sets the verbosity level, valid arguments are 0-3, default is 1
  ...
  --detpoly, -dp
     use the brand-new deterministic polytime algorithm
\end{verbatim}
Finally, calling \texttt{bin/pgsolver -dp} or \texttt{bin/pgsolver --detpoly} will solve a parity game
using your algorithm.

We leave it up to you to find the right bits in the code that you have to edit in order to make
\texttt{pgsolver} say that you have also contributed to the tool.



\section{Useful Functions}

When designing a new solver you may need some functionality that other solvers rely on as well. In that
case there is a good chance that one of the modules already contains the function you are looking for,
and you do not have to re-implement it. In the following we describe some of the implemented functions
that are most likely to be useful for solving parity games in general. All path names are relative to
the subdirectory \texttt{src}, i.e.\ when we speak about module \texttt{Basics} in the \texttt{paritygame}
directory for example then this can be found in the file \texttt{src/paritygame/basics.ml}.

%\paragraph{Basics Module}
\subsection{The {\ttfamily Basics} Module}

This module in the subdirectory \texttt{utils} contains only one function at the moment which is used to
output string messages to \texttt{STDOUT} depending on the configured verbosity level.

\begin{description}
\itemsep3mm
\item \verb+val message : int -> (unit -> string) -> unit+ \ \\
Calling \verb+message v (fun _ -> s)+ outputs the string \verb+s+ on \texttt{STDOUT} if the currently
set verbosity level is greater or equal than \verb+v+.

\item \verb+val verbosity_level_verbose : verbosity_level+ \ \\
Predefined verbosity level constant for verbose output (2).

\item \verb+val verbosity_level_default : verbosity_level+ \ \\
Predefined verbosity level constant for debug output (3).
\end{description}


\subsection{The {\ttfamily Paritygame} Module}

This module in the \texttt{paritygame} subdirectory contains many useful routines for creating, accessing,
manipulating, printing, decomposing and generally handling parity games. Some of the most useful functions
are as follows:

\begin{description}
\itemsep3mm
\item \verb+val print_game : paritygame -> unit+ \ \\
Calling \verb+print_game game+ prints game on \texttt{STDOUT} s.t.\ it could be parsed again.

\item \verb+val pg_max_prio : paritygame -> int+ \ \\
Calling \verb+pg_max_prio game+ returns the greatest priority occurring in the \verb+game+.

\item \verb+val pg_min_prio : paritygame -> int+ \ \\
Calling \verb+pg_min_prio game+ returns the least priority occurring in the \verb+game+.

\item \verb+val pg_max_prio_for : paritygame -> int -> int+ \ \\
Calling \verb+pg_max_prio_for game player+ returns the greatest reward for \verb+player+ occurring in the \verb+game+.

\item \verb+val pg_get_index : paritygame -> int+ \ \\
Calling \verb+pg_get_index game+ returns the index of the \verb+game+.

\item \verb+val pg_remove_nodes : paritygame -> int list -> unit+ \ \\
Calling \verb+pg_remove_nodes game node_list+ removes all nodes specified in \verb+node_list+.

\item \verb+val pg_remove_edges : paritygame -> (int * int) list -> unit+ \ \\
Calling \verb+pg_remove_edges game edge_list+ removes all edges specified in \verb+edge_list+.

\item \verb+val collect_max_prio_nodes : paritygame -> int list+ \ \\
Calling \verb+collect_max_prio_nodes game+ returns all nodes with greatest priority.

\item \verb+val subgame_by_list : paritygame -> int list -> paritygame+ \ \\
Calling \verb+subgame_by_list game nodes+ returns a compressed subgame induced and ordered by the \verb+nodes+-list

\item \verb+val merge_strategies_inplace : strategy -> strategy -> unit+ \ \\
Calling \verb+merge_strategies_inplace strat1 strat2+ adds all strategy decisions from \verb+strat2+ to
\verb+strat1+. Throws an \verb+Unmergable+-Exception if the domain of both strategies is not empty, i.e.\ if
there is an index $i$ such that the values of \verb+strat1+ and \verb+strat2+ at index $i$ are both not
$-1$.

\item \verb+val merge_solutions_inplace : solution -> solution -> unit+ \ \\
Calling \verb+merge_solutions_inplace sol1 sol2+ adds all solution informations from \verb+sol2+ to \verb+sol1+.
Throws an \verb+Unmergable+-Exception if the domain of both solutions is not empty, see above.

\item \verb+val game_to_transposed_graph : paritygame -> int list array+ \ \\
Calling \verb+game_to_transposed_graph game+ returns the transposed graph associated with the \verb+game+. It is
stipped of information about nodes' owners and priorities. At index $i$ it contains a list containing the
identifiers of all predecessors of node $i$ in \verb+game+.

\item \verb+val game_to_graph : paritygame -> int list array+ \ \\
Calling \verb+game_to_graph game+ returns the graph associated with the \verb+game+. See the former function
for a description of how graphs are represented.

\item
\begin{verbatim}
val strongly_connected_components: paritygame ->
                                   int list array *
                                   int array *
                                   int list array *
                                   int list
\end{verbatim}
Calling \verb+strongly_connected_components game+ decomposes the \verb+game+ into its SCCs. It returns a tuple
\verb+(sccs, sccindex, topology, roots)+ where \verb+sccs+ is an array mapping each SCC to its list of nodes,
\verb+sccindex+ is an array mapping each node to its SCC (represented by an integer value), \verb+topology+ is an
array mapping each SCC to the list of its immediate successing SCCs and \verb+roots+ is the list of SCC having no
predecessing SCCs.

\item
\begin{verbatim}
val attr_closure_inplace: paritygame -> strategy ->
                          int -> int list -> int list
\end{verbatim}
Calling \verb+attr_closure_inplace game strategy player region returns+ the attractor for the given \verb+player+
and \verb+region+. Additionally all necessary strategy decisions for \verb+player+ leading into the \verb+region+
are added to \verb+strategy+.

\end{description}


\subsection{The {\ttfamily Univsolve} Module}

This module in the \texttt{paritygame} subdirectory contains the universal solver and some possibly helpful functions
for the universal solving process:
\begin{description}
\itemsep3mm

\item
\begin{verbatim}
val universal_solve :
  universal_solve_options -> (paritygame -> solution * strategy) ->
      paritygame -> solution * strategy
\end{verbatim}
Calling \verb+universal_solve verbosity solver game+ starts the universal solving process using \verb+solver+ as a
backend. It returns the solved game as a pair of \verb+solution+ and \verb+strategy+.

\item
\begin{verbatim}
val universal_solve_by_player_solver :
  universal_solve_options ->
      (paritygame -> int -> solution * strategy) ->
      paritygame -> solution * strategy
\end{verbatim}
Calling \verb+universal_solve_by_player_solver verbosity player_solver game+ \linebreak starts the universal
solving process using \verb+player_solver+ as a backend. Instead of a solver backend that solves arbitrary SCCs
\verb+player_solver+ is supposed to solve an SCC w.r.t.\ a given player, for instance if \verb+player_solver scc 0+
is called, the backend is supposed to determine whether a node is won by player $0$ (return $0$ for this node) or
not (return $-1$ for this node). The strategy that is returned by \verb+player_solver+ is only considered w.r.t.\
the given player. The function returns the solved game as a pair of \verb+solution+ and \verb+strategy+.

\item
\begin{verbatim}
val universal_solve_trivial :
   verbosity_level -> paritygame -> solution * strategy
\end{verbatim}
Calling \verb+universal_solve_trivial verbosity game+ starts the universal solving process with a trivial i.e.\
not-solving backend. This function is legal to be called when \verb+game+ can be completely solved by the universal
solving process without requiring the backend. Calling \verb+universal_solve_trivial+ returns the solved game as a
pair of \verb+solution+ and \verb+strategy+.

\item
\begin{verbatim}
val compute_winning_nodes :
   verbosity_level -> paritygame -> strategy -> int -> int list
\end{verbatim}
Calling \verb+compute_winning_nodes verbosity game strategy player+ considers \linebreak the subgame of \verb+game+ w.r.t.\
the \verb+strategy+ decisions for \verb+player+; the strategy is assumed to be total w.r.t.\ \verb+player+. It
returns the list of nodes \verb+player+ wins on the game following \verb+strategy+.
\end{description}
In all these cases the argument \verb+verbosity+ is used to determine whether or not statistics should be printed
at the end.

%%% Local Variables:
%%% mode: latex
%%% TeX-master: "main"
%%% End:
