\label{chp:devguide}

The purpose of this chapter is to provide enough insight into the structure
of the \pgsolver tool such that it becomes possible to extend it with a new algorithm.
We assume reasonable OCaml programming skills and some familiarity with parity games, though.

Implementing a new algorithm requires two steps only.
\begin{enumerate}
\item Create an OCaml source code file that contains the solver. This is just a function
      which takes a parity game and returns a partition of (a subset of) its node set into
      winning regions and winning strategies for the two players. It must reside in a new
      module.
\item Integrate this function into the main program such that the new solver can be used
      by giving \texttt{pgsolver} a certain command-line argument. Then adjust the
      \texttt{SolverList} such that the new module is compiled with the rest of the tool.
\end{enumerate}


\section{Structure of the Source Code}

In the following, all references to files are given relatively to the directory into which
\pgsolver was unpacked during the installation process. There, the subdirectory \texttt{src}
is the important one for this step. It contains the source files of all the modules that
make up the \pgsolver library. They are organised into the following subdirectories respectively.
Note that, when we speak of \emph{modules} we often refer to files only that implicitly count as
a module in OCaml.
\begin{description}
\item[\texttt{generators}] contains modules that make up the benchmark generators;
\item[\texttt{tools}] contains tools for the creation or manipulation of parity games;
\item[\texttt{paritygame}] contains modules that define parity games as a data structure and
     provide functionality for that, e.g.\ in the form of procedures that compute SCC decompositions
     etc.;
\item[\texttt{pgsolver}] contains modules for the actual program \texttt{pgsolver};
\item[\texttt{solvers}] contains the parity game solvers, each in a separate module.
\end{description}

\section{Data Types and Functions for Manipulating Parity Games}

The signature file
\begin{verbatim}
src/paritygame/paritygame.mli
\end{verbatim}
contains the definition of parity games as an abstract data structure.
A parity game is internally stored as an array of node definitions but this may or may not change in the future
in favour of a more efficient implementation. There are also (abstract) data types for representing nodes, players,
priorities, sets of nodes, strategies, solutions and solutions, which will be introduced in detail in the following.

\subsection{Nodes}

The type of a node in a paritygame is one of the few types that are not abstract. For sake of efficiency, a node is 
simply an integer value:
\begin{verbatim}
   type node = int
\end{verbatim}
Beware that the type node may become abstract in the future, so try to avoid to arithmetic etc.\ on nodes. 

There is currently only one function that is specific to nodes; it turns a node into a string for output purposes.
\begin{verbatim}
   val nd_show : node -> string
\end{verbatim}


\subsection{Sets of Nodes}

Many algorithms manipulate sets of nodes, for instance they take the set of all nodes with highest priority etc. For 
such purposes, \verb#paritygame.mli# provides an abstract data type and functions for storing and manipulating sets of nodes.
\begin{verbatim}
   type nodeset
\end{verbatim}
Currently, this type is implemented as a simple ordered list without duplicates. The functions for creation and manipulation
follow the standard \verb#Set# interface. Use
\begin{verbatim}
   val ns_empty   : nodeset
   val ns_make    : node list -> nodeset
\end{verbatim}
to create a set of nodes, either initially empty or from a given list of nodes. It is possible to add nodes to existing sets
or deletes nodes from them using
\begin{verbatim}
   val ns_add     : node -> nodeset -> nodeset
   val ns_del     : node -> nodeset -> nodeset
\end{verbatim}
In the latter case \verb#ns_del v vs# returns vs if \verb#v# is not a member of \verb#vs#.

Desctruction of node sets is implemented as 
\begin{verbatim}
   val ns_nodes   : nodeset -> node list
\end{verbatim}
which returns all the elements of a set of nodes as a list.

For tests on emptiness and on membership of a particular node in a set use
\begin{verbatim}
   val ns_isEmpty : nodeset -> bool
   val ns_elem    : node -> nodeset -> bool
\end{verbatim}
There are more general tests, for instance if a node set includes some node with a particular property, or if all of them do so:
\begin{verbatim}
   val ns_exists  : (node -> bool) -> nodeset -> bool			
   val ns_forall  : (node -> bool) -> nodeset -> bool			
\end{verbatim}
It is possible to extract nodes from a node set, either some with a particular property (\verb#ns_find#) or a random one (\verb#ns_some#)
or the first, resp.\ last in some canonical order (\verb#ns_first#, \verb#ns_last#), or the one that is maximal w.r.t.\ a total order
(\verb#ns_max#).
\begin{verbatim} 
   val ns_find    : (node -> bool) -> nodeset -> node
   val ns_some    : nodeset -> node
   val ns_first   : nodeset -> node
   val ns_last    : nodeset -> node
   val ns_max     : nodeset -> (node -> node -> bool) -> node
\end{verbatim}
The cardinality of a set of nodes is computed by
\begin{verbatim}
   val ns_size    : nodeset -> int
\end{verbatim}
Finally, there are the typical iterator functions 
\begin{verbatim}
   val ns_fold    : ('a -> node -> 'a) -> 'a -> nodeset -> 'a
   val ns_iter    : (node -> unit) -> nodeset -> unit
   val ns_map     : (node -> node) -> nodeset -> nodeset
   val ns_filter  : (node -> bool) -> nodeset -> nodeset
\end{verbatim}
that behave as one would expect. Note that there is no control over the order in which nodes in a set are processed in \verb#ns_iter f vs# 
or \verb#ns_fold f x vs#; hence for a uniquely determined result the function \verb#f# should be independent of that order. 


\subsection{Players and Priorities}

There are two types representing players and priorities.
\begin{verbatim}
   type player 
   type priority = int
\end{verbatim}
Players are abstract but priorities are just integers. Only non-negative integers should be used as priorities.

There are constants defining the two players of a parity game. Some algorithms may have to defer the decisio about who wins which node in a game;
for such cases there is also a constant representing an undefined player. At last, there is a function which can be used to obtain a random player
with even distribution between \verb#plr_Even# and \verb#plr_Odd#.
\begin{verbatim} 
   val plr_Even   : player
   val plr_Odd    : player
   val plr_undef  : player
   val plr_random : unit -> player
\end{verbatim}
There are a few functions realising some minor arithmetic on the data type \verb#player#, for instance computing a player's opponent, for obtaining
a string representation, or for applying a function to both players.
\begin{verbatim}
   val plr_opponent : player -> player
   val plr_show     : player -> string
   val plr_iterate  : (player -> unit) -> unit
\end{verbatim}
There are some functions that link players and priorities, for instance by checking whether a priority is good for some player in the sense that
even priorities are good for \verb#plr_Even# and odd ones for \verb#plr_Odd#. Likewise it is possible to obtain the benefitting player directly.
\begin{verbatim}
   val prio_good_for_player : priority -> player -> bool
   val plr_benefits         : priority -> player
\end{verbatim}
These functions essentially test the parity of an integer value; there are also functions to do this directly. 					
\begin{verbatim}
   val odd: priority -> bool
   val even: priority -> bool
\end{verbatim}
It is unlikely, yet not impossible, that the type \verb#priority# will become abstract in the future. Hence, it is advisable to only manipulate
priorities through these functions.


\subsection{Parity Games}

The most important type in this module is the data type for parity games.
\begin{verbatim}
   type paritygame
\end{verbatim}
Note that as of version 4.0, the implementation of this type is no longer visible to the other modules, for reasons of code reliability. There are,
however, several functions that allow parity games to be created, manipulated and used.

There are three ways for creating a fresh parity game. 
\begin{verbatim}
   val pg_create : int -> paritygame
   val pg_init   : int -> 
                   (node -> (priority * player * node list * string option)) -> 
                   paritygame
   val pg_copy   : paritygame -> paritygame
\end{verbatim}
The former two expect the number of nodes of the game as a parameter. In case of \verb#pg_create#, the resulting game has no edges, the owner of each node
is \verb#plr_undef# and the priorities cannot be assumed to be non-negative integer values. The second function can be used to create a game and
fill it with some structure. Calling \verb#pg_init n f# will return a parity game with \verb#n# nodes such that the result of calling \verb#f i#
contains the priority, owner, list of successor nodes and possibly a name of the \verb#i#-th node in the game. Function \verb#pg_copy# creates a fresh
copy of an existing game which may be handy for algorithms that manipulate games and have to revert certain changes.

Given a parity game, such information about a node can be extracted using the following functions.
\begin{verbatim}
   val pg_get_priority     : paritygame -> node -> priority
   val pg_get_owner        : paritygame -> node -> player
   val pg_get_successors   : paritygame -> node -> nodeset
   val pg_get_predecessors : paritygame -> node -> nodeset
\end{verbatim}
It is possible to search for the node name (as an integer, resp.\ value of type \verb#node#) given its description.
\begin{verbatim}
   val pg_find_desc  : paritygame -> string option -> node
\end{verbatim}
There are two function for finding nodes according to their priorities.
\begin{verbatim}
   val pg_prio_nodes              : paritygame -> priority -> node list
   val pg_get_selected_priorities : paritygame -> 
                                    (priority -> bool) -> 
                                    priority list
\end{verbatim}
A parity game can be modified inplace by setting, resp.\ changing the priority or string description or owner of a node, or by adding an 
edge between two nodes.
\begin{verbatim}
   val pg_set_priority : paritygame -> node -> priority -> unit
   val pg_set_owner    : paritygame -> node -> player -> unit
   val pg_get_desc     : paritygame -> node -> string option
   val pg_set_desc     : paritygame -> node -> string option -> unit
   val pg_get_desc'    : paritygame -> node -> string
   val pg_set_desc'    : paritygame -> node -> string -> unit
   val pg_add_edge     : paritygame -> node -> node -> unit
\end{verbatim}
It is also possible to delete a set of nodes given as a list, or to remove a single edge or a set of edges, given as a list of node pairs. These
deletions also work inplace.
\begin{verbatim}
   val pg_remove_nodes : paritygame -> node list -> unit
   val pg_del_edge     : paritygame -> node -> node -> unit
   val pg_remove_edges : paritygame -> (node * node) list -> unit
\end{verbatim}
Note that using any of these can result in a game in which a node has no successor anymore. Deleting undefined nodes or non-existing edges results
in no change.

Some structural information about a parity game can be obtained, namely the size (i.e.\ number of nodes), the number of edges, or the index (i.e.\ number
of different priorities). 
\begin{verbatim}
   val pg_size	         : paritygame -> int
   val pg_node_count     : paritygame -> int
   val pg_edge_count     : paritygame -> int
   val pg_get_index      : paritygame -> int
   val pg_get_priorities : paritygame -> priority list
\end{verbatim}
The latter returns a list of all the priorities that are being used in the game.

There is a subtle difference between \verb#pg_size# and \verb#pg_node_count#. The latter only counts nodes that are defined, so for instance
\verb#pg_size (pg_create 2)# will return \verb#2#, while \verb#pg_node_count (pg_create 2)# will return \verb#0#. Checking whether a node is defined
can be done using
\begin{verbatim}
   val pg_isDefined : paritygame -> node -> bool
\end{verbatim}
Finally, there are iterator functions for parity games that traverse through all defined nodes, resp.\ all existing edges.
\begin{verbatim}
  val pg_iterate      : (node -> 
                         (priority * player * nodeset * nodeset * string option) -> 
                         unit) -> paritygame -> unit
  val pg_edge_iterate : (node -> node -> unit) -> paritygame -> unit
\end{verbatim}

\subsection{Solutions and Strategies}

A solution of a parity game is a mapping of nodes to players, determining who wins the game starting in that particular node.
A (memory-less) strategy (for both players) is a set of edges, including exactly one edge for each node that is won by the
player who owns the node. The interface \verb#paritygame.mli# provides two data structures for solutions and strategies which are
based on the fact that nodes are non-negative integer values.
\begin{verbatim}
   type solution = player array
   type strategy = node array
\end{verbatim}
A solution with value \verb#plr_Even# at position \verb#v# for instance represents the fact that player Even wins node \verb#v#.
Such arrays can contain the value \verb#plr_undef#.

An array of type \verb#strategy# can contain the value \verb#nd_undef#, representing the fact that the strategy decision at that position
is either unknown or unnecessary because the node is not won by its owner.

Note that an array of type \verb#strategy# contains, in effect, two positional strategies -- one for each player.

There is no mechanism that links a solution or strategy array to a particular parity game. It is the programmer's task to ensure
that the solution for one game is not used for another game. This could result in runtime errors like array access outside their
bounds.

\section{Implementing a New Solver}

We will exemplarily describe how to implement and integrate a new algorithm that solves parity games. There
is more than one way to do so but there is a unique easiest way which we will follow here.

Suppose you have finally come up with a deterministic polynomial time algorithm. Surely this cannot miss in
the \pgsolver library. It is also quite probable that the algorithm is complex -- for otherwise someone
else would have found it before. Therefore you cannot wait for us to implement it. By the time we have
understood all the algorithm's subtleties our programming skills will have bitten the dust. In that case you
better do it yourself.

Find an expressive name for your algorithm that will be used to create a module and also later a command-line
parameter for the \texttt{pgsolver} program. Say \emph{Deterministic Polynomial Time Solver} was a good name.
Then create two files in the \texttt{src/solvers} subdirectory, for example called
\begin{verbatim}
detpoly.ml
detpoly.mli
\end{verbatim}
The signature file \texttt{detpoly.mli} must look like this.
\begin{verbatim}
val solve : Paritygame.paritygame ->
              Paritygame.solution * Paritygame.strategy
\end{verbatim}
The name of the function is actually irrelevant, but calling it \texttt{solve} is not a bad idea. Its type
is mandatory though. It must take a parity game and return a pair of a solution and a strategy using the
data types described above.

Now sit down and hack the code for the implementation of your solver into the file \texttt{detpoly.ml}. In the
simplest form this will look like
\begin{verbatim}
open Paritygame ;;
let solve game = ...
\end{verbatim}
However, note that this function will be called in the main program, and as the argument \texttt{game} it will
receive the parity game from the input. Hence, this does not automatically make use of the universal solver
described in Sect.~\ref{sec:universal}. This is because one cannot guarantee that the universal optimisations
compression, SCC decomposition and solving of special cases speed up \emph{any} solver.

If your algorithm turns out to be that quick you might not want the universal optimisations in which case you
can skip the following description of how to employ the universal solver with your new algorithm as a backend
and continue with the next section.

So you are still with us -- thanks for the trust you have in our universal solver! Now, you will still have
to implement your algorithm in a function of type
\texttt{Paritygame.paritygame -> Paritygame.solution * Paritygame.strategy} but it may be better to choose
a different name. For instance, create a function
\begin{verbatim}
let my_solver game = ...
\end{verbatim}
in \texttt{detpoly.ml} that contains the implementation of your solver. This may now assume that the argument
\texttt{game} consists of a single SCC only such that the priorities of at least two nodes have different
parities and it is not just one player who has real choices in all the nodes in this game.

Now all you have to do is to make sure that the \texttt{solve}-function declared in \texttt{detpoly.mli}
calls the universal solver using your solver as a backend. This is easily done, \texttt{detpoly.ml} should
look like this.
\begin{verbatim}
let solve = Univsolve.universal_solve
             (Univsolve.universal_solve_init_options_verbose
              !Univsolve.universal_solve_global_options)
              my_solver
\end{verbatim}
Note that, if your solver implemented in \verb#my_solver# is recursive, it can also call \verb#solve# instead
of \verb#my_solver# recursively in order to have the universal optimisations done in every step.


\section{Integrating the New Solver}

Continuing with the example of the previous section we assume that you have implemented a solver in the file
\texttt{src/solvers/detpoly.ml}. In particular, this file contains a function \texttt{solve} of type
\texttt{Paritygame.paritygame -> Paritygame.solution * Paritygame.strategy} which is declared in
\texttt{src/solvers/detpoly.mli}.

You will have to ensure that your file does automatically get compiled when \pgsolver is being built using
the \texttt{make} command. In \texttt{SolverList} you will find a variable declaration listing all the modules
that make up the entire program. It should look like this.
\begin{verbatim}
PGSOLVERSLIST=
	obj/recursive.cmx \
	obj/stratimprovement.cmx \
	...
	obj/stratimprlocal.cmx \
	obj/stratimprdisc.cmx 
\end{verbatim}
Append to this list \texttt{obj/detpoly.cmx}. It need not be at the end of this list but you have to make
sure that it occurs behind any other module that contains code which is used by your solver. This could
be the \texttt{paritygame} and \texttt{univsolve} part for example and will most certainly be the
\texttt{solvers} part. Now running \texttt{make} should compile the \texttt{detpoly}-module as well.

At last, you have to integrate your algorithm into the program so that it can be used through the
\texttt{pgsolver} program when given a certain command-line option. This could not be an easier. Add to
your file \texttt{detpoly.ml} the following code after your \texttt{solve} function.
\begin{verbatim}
let _ = Solvers.register_solver
          solve
          "detpoly"
          "dp"
          "use the brand-new deterministic polytime algorithm"
\end{verbatim}
The function \texttt{register\_solver} from the module \texttt{solvers.ml} does everything for you. You
simply have to give it four arguments.
\begin{itemize}
\item The first one, here \texttt{solve} is the function of the type described above that implements
      your algorithm.
\item The second one is a string which should contain a concise but reasonably expressive synonym for your
      algorithm. It must not contain white spaces because it will be prefixed by two dashes ``\texttt{--}''
      and used as a command-line parameter that tells \texttt{pgsolver} to use your algorithm.
\item The third one is a string which contains an even shorter synonym for your algorithm. It will be
      prefixed by a single dash symbol ``\texttt{-}'' and used as a command-line parameter for \texttt{pgsolver}
      with the same effect.
\item The fourth argument is again a string which contains a description of what \texttt{pgsolver} does
      when given any of the the command-line parameters derived from the second or third argument.
\end{itemize}
After compilation, called \texttt{bin/pgsolver --help} should result in output containing the following
lines.
\begin{verbatim}
PGSolver Collection Ver. 3: Parity Game Solver
Authors: Oliver Friedmann and Martin Lange, University of Munich,
         2008-2010
http://www.tcs.ifi.lmu.de/pgsolver

Usage: pgsolver [options] [infile]
Solves the parity game given in <infile>. If this argument is omitted it
reads a game from STDIN.

Options are
  -v <level>
     sets the verbosity level, valid arguments are 0-3, default is 1
  ...
  --detpoly, -dp
     use the brand-new deterministic polytime algorithm
\end{verbatim}
Finally, calling \texttt{bin/pgsolver -dp} or \texttt{bin/pgsolver --detpoly} will solve a parity game
using your algorithm.

We leave it up to you to find the right bits in the code that you have to edit in order to make
\texttt{pgsolver} say that you have also contributed to the tool.



\section{Useful Functions}

When designing a new solver you may need some functionality that other solvers rely on as well. In that
case there is a good chance that one of the modules already contains the function you are looking for,
and you do not have to re-implement it. In the following we describe some of the implemented functions
that are most likely to be useful for solving parity games in general. All path names are relative to
the subdirectory \texttt{src}, i.e.\ when we speak about module \texttt{Basics} in the \texttt{paritygame}
directory for example then this can be found in the file \texttt{src/paritygame/basics.ml}.

%\paragraph{Basics Module}
\subsection{The {\ttfamily Basics} Module}

This module in the subdirectory \texttt{utils} contains only one function at the moment which is used to
output string messages to \texttt{STDOUT} depending on the configured verbosity level.

\begin{description}
\itemsep3mm
\item \verb+val message : int -> (unit -> string) -> unit+ \ \\
Calling \verb+message v (fun _ -> s)+ outputs the string \verb+s+ on \texttt{STDOUT} if the currently
set verbosity level is greater or equal than \verb+v+.

\item \verb+val verbosity_level_verbose : verbosity_level+ \ \\
Predefined verbosity level constant for verbose output (2).

\item \verb+val verbosity_level_default : verbosity_level+ \ \\
Predefined verbosity level constant for debug output (3).
\end{description}


\subsection{The {\ttfamily Paritygame} Module}

This module in the \texttt{paritygame} subdirectory contains many useful routines for creating, accessing,
manipulating, printing, decomposing and generally handling parity games. Some of the most useful functions
are as follows:

\begin{description}
\itemsep3mm
\item \verb+val print_game : paritygame -> unit+ \ \\
Calling \verb+print_game game+ prints game on \texttt{STDOUT} s.t.\ it could be parsed again.

\item \verb+val pg_max_prio : paritygame -> int+ \ \\
Calling \verb+pg_max_prio game+ returns the greatest priority occurring in the \verb+game+.

\item \verb+val pg_min_prio : paritygame -> int+ \ \\
Calling \verb+pg_min_prio game+ returns the least priority occurring in the \verb+game+.

\item \verb+val pg_max_prio_for : paritygame -> int -> int+ \ \\
Calling \verb+pg_max_prio_for game player+ returns the greatest reward for \verb+player+ occurring in the \verb+game+.

\item \verb+val pg_get_index : paritygame -> int+ \ \\
Calling \verb+pg_get_index game+ returns the index of the \verb+game+.

\item \verb+val pg_remove_nodes : paritygame -> int list -> unit+ \ \\
Calling \verb+pg_remove_nodes game node_list+ removes all nodes specified in \verb+node_list+.

\item \verb+val pg_remove_edges : paritygame -> (int * int) list -> unit+ \ \\
Calling \verb+pg_remove_edges game edge_list+ removes all edges specified in \verb+edge_list+.

\item \verb+val collect_max_prio_nodes : paritygame -> int list+ \ \\
Calling \verb+collect_max_prio_nodes game+ returns all nodes with greatest priority.

\item \verb+val subgame_by_list : paritygame -> int list -> paritygame+ \ \\
Calling \verb+subgame_by_list game nodes+ returns a compressed subgame induced and ordered by the \verb+nodes+-list

\item \verb+val merge_strategies_inplace : strategy -> strategy -> unit+ \ \\
Calling \verb+merge_strategies_inplace strat1 strat2+ adds all strategy decisions from \verb+strat2+ to
\verb+strat1+. Throws an \verb+Unmergable+-Exception if the domain of both strategies is not empty, i.e.\ if
there is an index $i$ such that the values of \verb+strat1+ and \verb+strat2+ at index $i$ are both not
$-1$.

\item \verb+val merge_solutions_inplace : solution -> solution -> unit+ \ \\
Calling \verb+merge_solutions_inplace sol1 sol2+ adds all solution informations from \verb+sol2+ to \verb+sol1+.
Throws an \verb+Unmergable+-Exception if the domain of both solutions is not empty, see above.

\item \verb+val game_to_transposed_graph : paritygame -> int list array+ \ \\
Calling \verb+game_to_transposed_graph game+ returns the transposed graph associated with the \verb+game+. It is
stipped of information about nodes' owners and priorities. At index $i$ it contains a list containing the
identifiers of all predecessors of node $i$ in \verb+game+.

\item \verb+val game_to_graph : paritygame -> int list array+ \ \\
Calling \verb+game_to_graph game+ returns the graph associated with the \verb+game+. See the former function
for a description of how graphs are represented.

\item
\begin{verbatim}
val strongly_connected_components: paritygame ->
                                   int list array *
                                   int array *
                                   int list array *
                                   int list
\end{verbatim}
Calling \verb+strongly_connected_components game+ decomposes the \verb+game+ into its SCCs. It returns a tuple
\verb+(sccs, sccindex, topology, roots)+ where \verb+sccs+ is an array mapping each SCC to its list of nodes,
\verb+sccindex+ is an array mapping each node to its SCC (represented by an integer value), \verb+topology+ is an
array mapping each SCC to the list of its immediate successing SCCs and \verb+roots+ is the list of SCC having no
predecessing SCCs.

\item
\begin{verbatim}
val attr_closure_inplace: paritygame -> strategy ->
                          int -> int list -> int list
\end{verbatim}
Calling \verb+attr_closure_inplace game strategy player region returns+ the attractor for the given \verb+player+
and \verb+region+. Additionally all necessary strategy decisions for \verb+player+ leading into the \verb+region+
are added to \verb+strategy+.

\end{description}


\subsection{The {\ttfamily Univsolve} Module}

This module in the \texttt{paritygame} subdirectory contains the universal solver and some possibly helpful functions
for the universal solving process:
\begin{description}
\itemsep3mm

\item
\begin{verbatim}
val universal_solve :
  universal_solve_options -> (paritygame -> solution * strategy) ->
      paritygame -> solution * strategy
\end{verbatim}
Calling \verb+universal_solve verbosity solver game+ starts the universal solving process using \verb+solver+ as a
backend. It returns the solved game as a pair of \verb+solution+ and \verb+strategy+.

\item
\begin{verbatim}
val universal_solve_by_player_solver :
  universal_solve_options ->
      (paritygame -> int -> solution * strategy) ->
      paritygame -> solution * strategy
\end{verbatim}
Calling \verb+universal_solve_by_player_solver verbosity player_solver game+ \linebreak starts the universal
solving process using \verb+player_solver+ as a backend. Instead of a solver backend that solves arbitrary SCCs
\verb+player_solver+ is supposed to solve an SCC w.r.t.\ a given player, for instance if \verb+player_solver scc 0+
is called, the backend is supposed to determine whether a node is won by player $0$ (return $0$ for this node) or
not (return $-1$ for this node). The strategy that is returned by \verb+player_solver+ is only considered w.r.t.\
the given player. The function returns the solved game as a pair of \verb+solution+ and \verb+strategy+.

\item
\begin{verbatim}
val universal_solve_trivial :
   verbosity_level -> paritygame -> solution * strategy
\end{verbatim}
Calling \verb+universal_solve_trivial verbosity game+ starts the universal solving process with a trivial i.e.\
not-solving backend. This function is legal to be called when \verb+game+ can be completely solved by the universal
solving process without requiring the backend. Calling \verb+universal_solve_trivial+ returns the solved game as a
pair of \verb+solution+ and \verb+strategy+.

\item
\begin{verbatim}
val compute_winning_nodes :
   verbosity_level -> paritygame -> strategy -> int -> int list
\end{verbatim}
Calling \verb+compute_winning_nodes verbosity game strategy player+ considers \linebreak the subgame of \verb+game+ w.r.t.\
the \verb+strategy+ decisions for \verb+player+; the strategy is assumed to be total w.r.t.\ \verb+player+. It
returns the list of nodes \verb+player+ wins on the game following \verb+strategy+.
\end{description}
In all these cases the argument \verb+verbosity+ is used to determine whether or not statistics should be printed
at the end.

%%% Local Variables:
%%% mode: latex
%%% TeX-master: "main"
%%% End:
